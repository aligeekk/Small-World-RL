%\section{Approach}
\label{sec:approach}

% Explain small world
As motivation, we look at the Kleinberg's analysis of the ``small
world phenomenon'' in social networks, defined to be exhibited when
individuals operating under a decentralised algorithm can transmit a
message from source to destination using a {\em short} path using only
local information such as the locations of their immediate
acquaintances. 

% Statement of Kleinberg's results
Consider a $k$-dimensional lattice of $n$ people \footnote{
Kleinberg's proofs were limited to the $2$-dimensional case, but were
extended to the $k$-dimensional case by Martel and Nyugen
\cite{Martel2004} }, wherein each person is connected to one
non-neighbour, according to the distribution $P_{r}( u, v )
\propto \| u-v \|^{-r}$, where $\|u-v\|$ is the graph distance between
nodes $u$ and $v$, and $r$ is a parameter. Kleinberg proves (a) that
when $r=0$, i.e. extra connections are uniformly distributed, any
decentralized algorithm will have an expected delivery time
exponential in $\tilde{d}$ (the shortest path length between $u$ and
$v$), and (b) when $r=k$, an algorithm can be constructed whose
expected delivery time is only polynomial of small degree in
$\tilde{d}$.

Similarly, we define an MDP with options to exhibit the small world
property when an agent can efficiently reach a state of {\em maximal
value} using only its local information. We construct a set of
`small-world options' which connect states in the state-interaction
graph according to $P_r$. By relating distance of two
states in the state space, and the difference in value of the two
states, we are able to prove that for a particular exponent($r$), the
expected number of {\em decisions} an agent will have to make to reach
a globally maximal value state will be poly-logarithmic in
$|\states|$. 
