%\section{Approach}
\label{sec:approach}

% Explain small world
To answer this question, we look at the analysis of the ``small world
phenomenon'' in social networks by Kleinberg. Kleinberg defines the phenomenon
to be exhibited when when individuals can {\em efficiently} transmit a message
from source to destination knowing only the locations of their immediate
acquaintances with a decentralised algorithm. 
% Statement of Kleinberg's results
Consider a $k$-dimensional lattice of $n$ people \footnote{ Kleinberg's proofs
were limited to the $2$-dimensional case. Martel and Nyugen \cite{Martel2004}
extended them to the $k$-dimensional case }, wherein each person is connected to
one other person outside his/her immediate neighbours, according to the
distribution $P_{r}( u, v ) \propto \dist(u,v)^{-r}$, where $\dist(u,v)$
is the lattice distance between nodes $u$ and $v$, and $r$ is a parameter.
Kleinberg proves (a) that when $r=0$, i.e. the extra connections are uniformly
distributed, any decentralized algorithm will have an expected delivery time
that is exponential in $\tilde{d}$ (the shortest path length between $u$ and
$v$), and (b) when $r=k$, an algorithm can be constructed whose expected
delivery time is only polynomial of small degree in $\tilde{d}$.

Similarly, we define an MDP with options to exhibit the small world property
when an agent can efficiently reach a state of {\em maximal value} using only
its' local information. Using a similarly constructed $k$-dimensional lattice
for $\states$, where two states are connected by primitive actions if they are
neighbours, and by an option with an optimal deterministic policy otherwise. By
relating the distance of two states in the state space, and the difference in
value of the two states, we are able to prove that the expected number of {\em
decisions} an agent will have to make to reach a maximal value state will be poly-logarithmic in
$|\states|$.
