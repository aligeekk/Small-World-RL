\section{Introduction}
\label{sec:intro}

% Slow learning in RL
In large domains, RL agents generally require a large number of samples to
learn a good policy. The options framework proposed by Sutton, Precup and Singh
provides extended actions for which a policy is already learnt, reducing the
complexity of the learning task, and generally making the learning task faster.
An open question in the options framework is discovering the options
themselves.  There has been substantial work to learn options, mainly focussed
around identifying ``bottleneck'' states, either empirically as in \cite{Stolle}, or
more recently, using graph theoretic methods like betweeness \cite{Simsek} or
graph partitions \cite{Simsek2005}.

We would like to test an alternative hypothesis; we memorise many actions,
not necessarily bottleneck ones, and put them together; based on their
necessity in solving problems these actions are either reinforced, or gradually
forgotten.  The actions could be of varying complexity, and it is intuitive to
expect that we probably learn a great deal more {\em simple} actions than
complex ones. In context of the options framework, the ``complex actions''
correspond to options.

Our proposed approach is to use randomly constructed options that create a
'short-cut' between states, forming a sort of 'small-world' in the domain. This
approach can be viewed as an extension of Kleinberg's popular model
\cite{Kleinberg} in the Social Network Analysis field, and we would like to note
that RL domains are very grid-like as well. The analogy is further motivated by
observing that the policy followed by an agent in the MDP framework is like
distributed search; we are interested in moving from our source state to the
destination (goal) state using only information available locally, i.e. the
value function. We leave addressing the dynamics of such random options, i.e.
when options are added or removed, as future work.

As part of our work, we would like to address the following interesting questions,
\begin{enumerate}
\item
    What is the ``distance'' analogue in the RL domain? How can the dimension
    be characterised?
\item
    How many such random edges need to be added in order to learn new problems
    quickly? How does shaping the number of ``long short-cuts'' affect
    performance? Can we use similar results as proposed in Kleinberg's original
    work?
\item
    Can we apply this technique to create sub-goals in ``local'' state transition
    graphs as in \cite{Simsek2005}?
\end{enumerate}

Finally, we in order to evaluate the performance of our approach, we will
present the learning performance of agents trained on the Tic-Tac-Toe, Taxi and
Playroom domains, using the intra-option and macro-Q learning algorithms. We
will compare our performance with the agent without options, as well as an
agent trained using the options learnt using betweeness measures. 

