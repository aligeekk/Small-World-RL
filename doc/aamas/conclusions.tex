\section{Conclusions and Future Work}
\label{sec:conclusions}

We have emperically shown that agents using options generating to make the
state-space a `small world' converge reliably to near-optimal behaviour for a
single domain. We clearly need to experiment on several other domains,
especially larger ones.

Our approach easily extends to stochastic domains, where the shortest path would
naturally extend to the optimal policy between the two points. It would also be
easy to find options based on a state space graph constructed from several
trajectories, though such an approach leads to interesting questions about the
dimensionality of the state space, and which value of $r$ that would need to be
used.

Another interesting direction to take this work forward would be to explore the
dynamics of these options; would it be possible to merge or remove options
dynamically? If so, it would imply this approach would scale very gracefully,
and might make for a good cognitive model.

% Small World options > random
% Small World options good enough

% Further experimentation
% Dimensionality
% Defining in stochastic domains
