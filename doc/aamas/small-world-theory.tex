\section{Small Worlds}
\label{sec:small-worlds}

% Introduction and motivation for the proof
In this section we will tackle the proof of the main theorem in
\secref{sec:theory},

\begin{theorem}
  %
  Let $f : V \to \Re$ be a function embedded on the graph $\graph(V,E)$,
  such that, $\kappa_1 \|u-v\| - c_1 \le \|f(u) - f(v)\| \le \kappa_2
  \|u - v\| - c_2$, where $0 \le \kappa_1 \le \kappa_2$, and $0 \le c_2
  \le \frac{c_1}{2}$. Let $M_f$ be the global maxima of $f$. Let
  \egreedyalgo be an $\epsilon$-greedy algorithm with respect to $f$,
  i.e.  an algorithm which chooses with probability $1-\epsilon$ to
  transit to the neighbouring state closest to $M_f$, i.e. $N(u)
  = \argmin_v \|f(v) - f(M_f)\|$.
  
  If $\graph(V,E)$ is $k$-dimensional lattice, and contains a long
  distance edge distributed according to $P_k: p(u,v) \propto
  \|u-v\|^{-k}$, then \egreedyalgo takes $O( (\log |V|)^2 )$ steps to
  reach $M_f$.
\end{theorem}
\begin{proof}

This result is a simple extension of Kleinberg's result in
\cite{Kleinberg2000}, and follows the proof presented there, albeit with
the somewhat cleaner notation and formalism of \cite{Martel2004}. We
begin by defining the necessary formalism to present the proof.

\begin{definition}
Let us define $\ball_l(u)$ to be the set of nodes contained within
a ``ball'' of radius $l$ centered at $u$, i.e.  $\ball_l(u) = \{ v \mid
\|u - v\| < l \}$, and $\sball_l(u)$ to be the set of nodes on its
surface, i.e. $\sball_l(u) = \{ v \mid \|u - v\| = l \}$.

Given a function $f:V \to \Re$ embedded on $\graph(V,E)$, we analogously
define $\ballf_l(u) = \{ v \mid \|f(u) - f(v)\| < l \}$. For notational
convenience, we take $\ballf_l$ to be $\ballf_l(M_f)$.
\end{definition}

\begin{lemma}
    The inverse normalised coefficient for $p(u,v)$ is $c_u = \theta(
    \log n )$, and $p(u,v) = \|u - v\|^{-k} \theta(\log n)^{-1}$.
\end{lemma}
\begin{proof}
    \begin{eqnarray*}
        c_u &=& \sum_{v \ne u} \|u - v\|^{-k} \\
            &=& \sum_{j=1}^{k(n-1)} \sball_j(u) j^{-k}.
    \end{eqnarray*}
    It can easily be shown that the $\sball_l(u) = \theta( l^{k-1} )$.
    Thus, $c_u$ reduces to a harmonic sum, and hence is equal to
    $\theta( \log n )$. The second part of the lemma follows as $p(u,v)
    = \frac{ \|u - v\|^{-k} }{c_u}$. 
\end{proof}

We are now ready to prove that \egreedyalgo takes $O( (\log |V|)^2 )$
decisions. Let a node $u$ be in phase $j$ when $u \in \ballf_{2^{j+1}}
\setminus \ballf_{2^{j}}$. The probability that phase $j$ will end this
step is equal to the probability that $N(u) \in \ballf_{2^{j}}$. 

The size of $\ballf_{2^{j}}$ is at least $|\ball_{\frac{
2^{j}+c_2}{\kappa_2}}| = \theta(\frac{2^{j}+c_2}{\kappa_2})$. The
distance between $u$ and a node in $\ballf_{2^{j}}$ is at most
$\frac{2^{j+1} + c_1}{ \kappa_1 } + \frac{2^{j} + c_2}{\kappa_2}
< 2(\frac{2^{j+1} + c_2}{\kappa_2})$. The probability of a link between
these two nodes is at least $(\frac{2^{j+2} + 2 c_1}{\kappa_1})^{-k}
\theta(\log n)^{-1} $. Thus, 

\begin{eqnarray*}
    P(u, \ballf_{2^{j}} ) &\ge& \frac{(1-\epsilon)}{\theta( \log n )} (\frac{2^{j}+c_2}{\kappa_2})^{k} \times (\frac{2^{j+2} + 2 c_1}{\kappa_1})^{-k} \\
    &\ge& \frac{(1-\epsilon)}{\theta( \log n )} \times (\frac{\kappa_1}{4\kappa_2} )^{k} \times ( \frac{ 1 + \frac{c_2}{2^{j}} }{ 1 + \frac{c_1}{2 \times 2^{j}} })^{k}\\
    &\ge& \frac{(1-\epsilon)}{\theta( \log n )} \times (\frac{\kappa_1}{4\kappa_2} )^{k} \times ( \frac{ 1 + c_2 }{ 1 + \frac{c_1}{2} })^{k} .\\
\end{eqnarray*}

Let number of decisions required to leave phase $j$ be $X_j$. Then, 
\begin{eqnarray*}
    \E[X_j] &\le& \sum_{i=0}^{\infty} (1 - P(u, \ballf_{2^{j}} ))^i \\
            &\le& \frac{1}{P(u, \ballf_{2^{j}} )} \\
            &\le& \theta( \log n ) \frac{1}{(1-\epsilon)} (\frac{4 \kappa_2}{\kappa_1})^{k} ( \frac{ 1 + \frac{c_1}{2} }{ 1 + c_2 })^{k}\\
            &\le& \theta( \log n ).
\end{eqnarray*}
Thus, it takes at most $O(\log n)$ decisions to leave phae $j$. By construction, there are at most $\log n$
phases, and thus at most $O((\log n)^2)$ decisions.
\end{proof}

