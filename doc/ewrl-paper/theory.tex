\section{Theory}
\label{sec:theory}

% Introduction and motivation for the proof

% Abstract result
Consider a $k$-dimensional lattice $\graph(V,E)$ with random links distributed
according to an inverse power law distribution $p(u,v) \propto \|u-v\|^{-k}$,
where $\|u - v\|$ is the Manhattan distance between $u$ and $v$. Let us define
$\ball_l(u)$ to be the set of nodes contained within a ``ball'' of radius $l$
centered at $u$, i.e.  $\ball_l(u) = \{ v \mid \|u - v\| < l \}$, and
$\sball_l(u)$ to be the set of nodes on its surface, i.e. $\sball_l(u) = \{ v
\mid \|u - v\| = l \}$.

\begin{lemma}
    The inverse normalised coefficient $c_u = O( \log(n) )$, and $p(u,v) =
    \|u - v\|^{-k} \Omega( \log(n)^{-1} )$.
\end{lemma}
\begin{proof}
    \begin{IEEEeqnarray*}{rCl}
        c_u &=& \sum_{v \ne u} \|u - v\|^{-k} \\
            &=& \sum_{j=1}^{k(n-1)} \sball_j(u) j^{-k}.
    \end{IEEEeqnarray*}
    It can easily be shown that the $\sball_l(u) = \theta( l^{k-1} )$. Thus, the
    $c_u$ reduces to a harmonic sum, and trivially, $c_u = \theta( \log(n) )$.
    The second part of the lemma follows as $p(u,v) = \frac{ \|u - v\|^{-k} }{c_u}$. 
    \\ \qed
\end{proof}

\begin{lemma}
    \label{lm:link-to-ball}
    For any $k \ge 1$, there exists a constant $\arbcnst$ such that for any ball
    $\ball_l(v)$ and node $u$ outside $\ball_l(v)$ on $\graph$ such that,
    $P(u,\sball_l(v)) \le \frac {\arbcnst}{m\log(n)}$, where $m = \min_{w \in
    \sball_l(v)} \|u-w\|$, the minimum distance from the surface of the ball.
    Further, if $k \ge 2$, and $l \le m$, then $P(u,\sball_l(v)) \le \frac
    {\arbcnst l}{m^2\log(n)}$. 
\end{lemma}
\begin{proof}
    Refer Fact 4 of \cite{Martel2004}.
    \\ \qed
\end{proof}

Consider a function $f$ embedded on $\graph(V,E)$, i.e. $f : V \to \Re$, with
the property that $\|f(u) - f(v)\| = \kappa\|u - v\| + c$, where $c \in [c_1,
c_2]$. We analogously define $\ballf_l(u) = \{ v \mid \|f(u) - f(v)\| < l \}$. 

\begin{lemma}
    \label{lm:link-to-ballf}
    For any $k \ge 1$, there exists a constant $c$ such that for any ball
    $\ballf_l(v)$ and node $u$ outside $\ballf_l(v)$ on $\graph$ such that,
    $P(u,\sballf_l(v)) \le \frac {\arbcnst}{m\log(n)}$, where $m = \min_{w \in
    \sballf_l(v)} \|u-w\|$, the minimum distance from the surface of the ball.
    Further, if $k \ge 2$, and $l \le m$, then $P(u,\sballf_l(v)) \le \frac
    {\arbcnst(\kappa l+c_1)}{m^2\log(n)}$. 
\end{lemma}
\begin{proof}
    Nodes in $\sballf_l(v)$ lie between $\ball_{\kappa l+c_1}(v)$ and
    $\ball_{\kappa l+c_2}(v)$, thus, $P(u,\sball_f(v)) \le  P(u,\sball_{\kappa l+c_1}(v))$.
    From \lmref{lm:link-to-ball}, when $k \ge 1$, $P(u,\sballf_l(v)) \le \frac
    {\arbcnst}{m\log(n)}$, and when $k \ge 2$, $P(u,\sballf_l(v)) \le \frac
    {\arbcnst(\kappa l+c_1)}{m^2\log(n)}$.
    \\ \qed
\end{proof}

% Describe problem / algo

Let $v$ be the global maxima of $f$, and $N(u)$ be the next node found by
\greedyalgo from $u$. Let $\delta_v(u) = \|f(u)-f(v)\| - \|f(N(u))-f(v)\|$. The following lemma
directly follows from \lmref{lm:link-to-ballf}.

\begin{lemma}
    \label{lm:rate}
    For each $k \ge 1$, there exists a constant $\arbcnst$ such that, for
    $\graph$, any two nodes $u$, $v$, and any integer $1 < m < \|f(u) - f(v)\|$,
    $\Pr[ \delta_v(u) = m ] \le \frac{\arbcnst}{m \log(n)} \times \min \{ 1,
    \frac{ \kappa\|f(u)-f(v)\| + c_1 - m }{m} \}$.
\end{lemma}
\begin{proof}
    When $\delta_v(u) = m$,  $N(u)$ lies on $b = \sballf_{\|f(u)-f(v)\| -
    m}(v)$, and thus, from \lmref{lm:link-to-ballf}, we have $P(u,b)$ to be
    bounded by either, $\frac {\arbcnst}{m\log(n)}$ or $\frac {\arbcnst(\kappa\|u -
    v\| + c_1 - m)}{m^2\log(n)}$.
    \\ \qed
\end{proof}


\begin{theorem}
    \label{thm:decisions}
    \greedyalgo takes $O( \log^2(n) )$ decisions.
\end{theorem}
\begin{proof}
    Let a node $u$ be in phase $j$ when $u \in \ballf_{2^{j+1}} \setminus
    \ballf_{2^{j}}$. The probability that phase $j$ will end this step is equal
    to the probability that $N(u) \in \ballf_{2^{j}}$. 
    
    The size of $\ballf_{2^{j}}$ is at least $|\ball_{\kappa 2^{j}+c_1}| = O(
    (\kappa 2^{j}+c_1)^{k-1} )$. The probability of an edge between $u$ and a
    node in $\ballf_{2^{j}}$ is at least $ (\kappa 2^{j+1}+c_2)^{-k}
    \log(n)^{-1} $. Thus, 
    \begin{IEEEeqnarray*}{rCl}
        P(u, \ballf_{2^{j}} ) &\ge& (\kappa 2^{j}+c_1)^{k-1} \times (\kappa 2^{j+1}+c_2)^{-k} \theta( \log(n) )^{-1}  \\
        &\ge& \frac{1}{2 \theta( \log(n) )} \times \frac{ (1 + \frac{c_1}{\kappa 2^{j}})^{k-1} }{ (1+\frac{c_2}{2\kappa 2^{j}})^{k} }.\\
    \end{IEEEeqnarray*}
    As $c_2 > 2 c_1$, 
    \begin{IEEEeqnarray*}{rCl}
        P(u, \ballf_{2^{j}} ) &\ge& \frac{1}{2 \theta( \log(n) )} \times (1 + \frac{c_1}{\kappa 2^{j}})^{-1} (\frac{1 + \frac{c_1}{\kappa}}{1+\frac{c_2}{2\kappa}})^{k} \\
        P(u, \ballf_{2^{j}} ) &\ge& \frac{1}{2 \theta( \log(n) ) (1 + \frac{c_1}{\kappa}) } (\frac{1 + \frac{c_1}{\kappa}}{1+\frac{c_2}{2\kappa}})^{k}.\\
    \end{IEEEeqnarray*}

    Let number of decisions required to leave phase $j$ be $X_j$. Then,
    \begin{IEEEeqnarray*}{rCl}
        \E[X_j] &\le& \sum_{i=0}^{\infty} (1 - P(u, \ballf_{2^{j}} ))^i \\
                &\le& \frac{1}{P(u, \ballf_{2^{j}} )} \\
                &\le& 2 \theta( \log(n) ) (1 + \frac{c_1}{\kappa}) (\frac{1 + \frac{c_2}{2\kappa}}{1+\frac{c_1}{\kappa}})^{k} \\
                &\le& \theta( \log(n) ).
    \end{IEEEeqnarray*}
    Thus, $\E[X_j]$ is $O(\log(n))$. By construction, there are at most $\log(n)$
    phases, and thus at most $O(\log^2(n))$ decisions.
    \\ \qed
\end{proof}
